\documentclass{article}

\usepackage{cwpuzzle}
\usepackage{listings}
\usepackage{color}
\lstset{frame=tb,
  language=C++,
  aboveskip=3mm,
  belowskip=3mm,
  showstringspaces=false,
  columns=flexible,
  basicstyle={\small\ttfamily},
  numbers=none,
  keywordstyle=\color{blue},
  breaklines=true,
  breakatwhitespace=true,
  tabsize=4
}

\begin{document}
\section*{Puzzles}

\subsection*{Crossword:}

\begin{Puzzle}{20}{20} %
|*    |*    |*    |*    |*    |*    |*    |*    |*    |*    |*    |*    |*    |[1]M |O    |N    |A    |[2]D |*    |*    |.
|*    |*    |*    |*    |*    |*    |*    |*    |[3]C |O    |B    |[4]O |L    |*    |*    |*    |*    |O    |*    |*    |.
|*    |*    |*    |*    |*    |*    |*    |*    |*    |*    |*    |B    |*    |*    |*    |*    |[5]R |U    |B    |Y    |.
|*    |*    |*    |*    |*    |*    |*    |*    |*    |*    |*    |J    |*    |*    |*    |*    |*    |B    |*    |*    |.
|*    |*    |*    |[6]C |A    |M    |E    |L    |C    |A    |S    |E    |*    |*    |*    |*    |*    |L    |*    |*    |.
|*    |*    |*    |*    |*    |*    |*    |*    |*    |*    |*    |[7]C |O    |M    |P    |I    |L    |E    |R    |*    |.
|*    |*    |*    |*    |*    |*    |*    |*    |*    |*    |*    |T    |*    |*    |*    |*    |*    |*    |*    |*    |.
|*    |*    |*    |*    |*    |*    |*    |*    |*    |*    |[8]L |O    |O    |P    |S    |*    |*    |*    |*    |*    |.
|*    |*    |*    |*    |*    |*    |*    |*    |*    |*    |*    |R    |*    |*    |*    |*    |*    |*    |*    |*    |.
|*    |*    |*    |*    |*    |*    |*    |*    |*    |*    |*    |I    |*    |*    |*    |*    |*    |*    |*    |*    |.
|*    |*    |*    |*    |*    |*    |*    |*    |*    |*    |*    |E    |*    |*    |*    |*    |*    |*    |*    |*    |.
|*    |*    |*    |*    |*    |*    |[9]V |*    |*    |*    |*    |N    |*    |*    |*    |*    |*    |*    |*    |*    |.
|*    |*    |*    |*    |[10]S|P    |A    |G    |H    |E    |T    |T    |I    |*    |*    |*    |*    |*    |*    |*    |.
|*    |*    |*    |*    |*    |*    |R    |*    |*    |*    |*    |E    |*    |*    |*    |*    |*    |*    |*    |*    |.
|[11]P|*    |*    |*    |*    |[12]L|I    |F    |o    |*    |*    |D    |*    |*    |*    |*    |*    |*    |*    |*    |.
|Y    |*    |[13]R|*    |*    |*    |A    |*    |*    |*    |*    |*    |*    |*    |*    |*    |*    |*    |*    |*    |.
|T    |*    |U    |*    |*    |*    |B    |*    |*    |*    |*    |*    |*    |*    |*    |*    |*    |*    |*    |*    |.
|[14]H|A    |S    |K    |E    |L    |L    |*    |*    |*    |*    |*    |*    |*    |*    |*    |*    |*    |*    |*    |.
|O    |*    |T    |*    |*    |*    |E    |*    |*    |*    |*    |*    |*    |*    |*    |*    |*    |*    |*    |*    |.
|N    |*    |*    |*    |*    |*    |*    |*    |*    |*    |*    |*    |*    |*    |*    |*    |*    |*    |*    |*    |.
\end{Puzzle}


\begin{PuzzleClues}{\textbf{Across}} %
\Clue{1}{MONAD}{Pythagorean divinity} %
\Clue{3}{COBOL}{Primarily used in finance systems} %
\Clue{5}{RUBY}{An object oriented shiny rock} %
\Clue{6}{CAMELCASE}{capitaliseAndRemoveAllSpaces} %
\Clue{7}{COMPILER}{Developed by Grace Hopper} %
\Clue{8}{LOOPS}{While, for, etc} %
\Clue{10}{SPAGHETTI}{Bad code can be compared to this food} %
\Clue{12}{LIFO}{Pez-like stack} %
\Clue{14}{HASKELL}{No side effects} %
\end{PuzzleClues} %

\begin{PuzzleClues}{\textbf{Down}}
\Clue{2}{DOUBLE}{Holds real numbers} %
\Clue{4}{OBJECTORIENTED}{e.g. Java} %
\Clue{9}{VARIABLE}{x} %
\Clue{11}{PYTHON}{Non venomous and kind to newcomers} %
\Clue{13}{RUST}{Mozilla Research design} %
\end{PuzzleClues}

\subsection*{Takuzu:}
The goal of this problem is to fill the grid with 1 and 0. 
Rules:
\begin{description}
\item[1.] There have to be the same number of 1 and 0 in each line.
\item[2.] No more than two cells in a row can contain the same digit.
\item[3.] Each row and each column have to be unique.
\end{description}

% Please insert image attached in email here. Sorry about not being able to get it to display!
\begin{table}[ht]
\centering
\begin{tabular}{|c|c|c|c|c|c|c|c|c|c|c|}
\hline 0 &   & 0 &   & 1 & 1 &   & 1 & 1 &   \\ 
\hline   &   &   &   &   &   & 0 &   & 1 &   \\ 
\hline   &   & 0 &   &   & 1 &   &   &   &   \\ 
\hline 0 &   &   &   & 1 &   &   &   & 1 & 1 \\ 
\hline   &   & 0 &   &   &   & 1 &   &   &   \\ 
\hline 0 & 1 &   & 1 &   &   &   &   &   &   \\ 
\hline   &   &   & 1 &   &   & 0 &   & 0 & 0 \\ 
\hline   &   &   &   &   &   &   & 1 &   &   \\ 
\hline 1 &   & 1 & 1 &   & 0 &   &   &   & 0 \\ 
\hline 1 &   & 1 & 1 &   & 0 &   & 0 & 0 &   \\
\hline
\end{tabular}
\end{table} 

Answer:
\begin{table}[ht]
\centering
\begin{tabular}{|c|c|c|c|c|c|c|c|c|c|c|}
\hline 0 & 1 & 0 & 0 & 1 & 1 & 0 & 1 & 1 & 0 \\ 
\hline 0 & 1 & 1 & 0 & 1 & 0 & 0 & 1 & 1 & 0 \\ 
\hline 1 & 0 & 0 & 1 & 0 & 1 & 1 & 0 & 0 & 1 \\ 
\hline 0 & 0 & 1 & 0 & 1 & 1 & 0 & 0 & 1 & 1 \\ 
\hline 1 & 1 & 0 & 0 & 1 & 0 & 1 & 1 & 0 & 0 \\ 
\hline 0 & 1 & 0 & 1 & 0 & 0 & 1 & 0 & 1 & 1 \\ 
\hline 1 & 0 & 1 & 1 & 0 & 1 & 0 & 1 & 0 & 0 \\ 
\hline 0 & 1 & 0 & 0 & 1 & 1 & 0 & 1 & 0 & 1 \\ 
\hline 1 & 0 & 1 & 1 & 0 & 0 & 1 & 0 & 1 & 0 \\ 
\hline 1 & 0 & 1 & 1 & 0 & 0 & 1 & 0 & 0 & 1 \\
\hline
\end{tabular}
\end{table} 


\subsection*{Brain teasers:}
\begin{description}
\item[A.] In the hexadecimal number system, numbers are represented using 16 different digits: 0,1,2,3,4,5,6,7,8,9,A,B,C,D,E,F. The hexadecimal number AF, when written in the decimal number system, equals 10x16+15=175. How would you write 1917 as a hexadecimal number?
[Solution: 77D]

\item[B.] What does the obfuscated program below write to stdout?
\begin{lstlisting} 
main() {
  long long P = 1,
            E = 2,
            T = 5,
            A = 61,
            L = 251,
            N = 3659,
            R = 271173410,
            G = 1479296389,
            x[] = { G * R * E * E * T , P * L * A * N * E * T };
  puts((char*)x);
}
\end{lstlisting}
[Answer: Hello World!]
\end{description}
\end{document}
