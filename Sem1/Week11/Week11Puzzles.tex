\documentclass{article}

\usepackage{cwpuzzle}

\begin{document}
\section*{Puzzles}

\subsection*{Crossword:}

\begin{Puzzle}{20}{20} %
|*    |*    |*    |*    |*    |[1]H |A    |M    |M    |I    |N    |G    |D    |I    |S    |T    |A    |N    |C    |[2]E |.
|*    |*    |*    |*    |*    |*    |*    |*    |*    |*    |*    |*    |*    |*    |*    |*    |*    |*    |*    |L    |.
|*    |*    |*    |*    |[3]B |*    |*    |*    |[4]H |*    |[5]B |O    |Y    |E    |R    |M    |O    |O    |R    |E    |.
|*    |[6]Z |*    |*    |I    |*    |*    |*    |I    |*    |*    |*    |*    |*    |*    |*    |*    |*    |*    |C    |.
|*    |A    |*    |*    |O    |*    |*    |*    |[7]D |A    |T    |A    |B    |A    |S    |[8]E |*    |*    |*    |T    |.
|*    |L    |*    |*    |P    |*    |*    |*    |D    |*    |*    |*    |*    |*    |*    |D    |*    |*    |*    |R    |.
|*    |G    |*    |*    |Y    |*    |*    |*    |E    |*    |*    |*    |*    |*    |*    |I    |*    |*    |*    |O    |.
|*    |O    |*    |*    |T    |*    |*    |*    |N    |*    |*    |[9]M |*    |*    |*    |T    |*    |*    |*    |P    |.
|*    |R    |*    |*    |H    |*    |*    |*    |M    |*    |*    |I    |*    |*    |*    |D    |*    |*    |*    |H    |.
|[10]M|I    |C    |R    |O    |A    |R    |R    |A    |Y    |*    |C    |*    |*    |*    |I    |*    |*    |*    |O    |.
|*    |T    |*    |*    |N    |*    |*    |*    |R    |*    |*    |R    |*    |[11]A|S    |S    |A    |Y    |*    |R    |.
|*    |H    |*    |*    |*    |*    |*    |*    |K    |*    |*    |O    |*    |*    |*    |T    |*    |*    |*    |E    |.
|*    |M    |*    |*    |*    |*    |[12]B|I    |O    |I    |N    |F    |O    |R    |M    |A    |T    |I    |C    |S    |.
|*    |*    |*    |*    |*    |*    |*    |*    |V    |*    |*    |L    |*    |*    |*    |N    |*    |*    |*    |I    |.
|*    |*    |*    |*    |*    |*    |*    |*    |*    |*    |*    |U    |*    |*    |*    |C    |*    |*    |*    |S    |.
|*    |*    |*    |*    |*    |*    |*    |[13]S|U    |F    |F    |I    |X    |T    |R    |E    |E    |S    |*    |*    |.
|*    |*    |*    |*    |*    |*    |*    |*    |*    |*    |*    |D    |*    |*    |*    |*    |*    |*    |*    |*    |.
|*    |*    |*    |*    |*    |*    |[14]D|A    |T    |A    |M    |I    |N    |I    |N    |G    |*    |*    |*    |*    |.
|*    |*    |*    |*    |*    |*    |*    |*    |*    |*    |*    |C    |*    |*    |*    |*    |*    |*    |*    |*    |.
|*    |*    |*    |*    |*    |*    |*    |*    |[15]S|U    |B    |S    |T    |R    |I    |N    |G    |*    |*    |*    |.
\end{Puzzle}


\begin{PuzzleClues}{\textbf{Across}} %
\Clue{1}{HAMMINGDISTANCE}{Edit distance, but with substitutions} %
\Clue{5}{BOYERMOORE}{String matching algorithm, not quite Knuth-Morris-Pratt} %
\Clue{7}{DATABASE}{Data storage, e.g. JSON formatted} %
\Clue{10}{MICROARRAY}{2D array, pretty damn small} %
\Clue{11}{ASSAY}{Method for measuring biological activity, only sounds like speech} %
\Clue{12}{BIOINFORMATICS}{Computational analysis of biological data} %
\Clue{13}{SUFFIXTREES}{Inverse of prefix trees} %
\Clue{14}{DATAMINING}{Querying large databases, not actually in a quarry} %
\Clue{15}{SUBSTRING}{Contiguous all the way} %
\end{PuzzleClues} %
%
\begin{PuzzleClues}{\textbf{Down}}
\Clue{2}{ELECTROPHORESIS}{Molecules are separated by size because of an electric field in this method} %
\Clue{3}{BIOPYTHON}{Closer to a real snake, but still code}
\Clue{4}{HIDDENMARKOV}{\_\_\_\_\_\_\_\_ model: statistical model for an ordered sequence of variables} %
\Clue{6}{ZALGORITHM}{Method of string matching, at the end of the alphabet} %
\Clue{8}{EDITDISTANCE}{Levenshtein distance with less Soviets} %
\Clue{9}{MICROFLUIDICS}{Chemical reactions on tiny scales} %

\end{PuzzleClues}

\subsection*{Takuzu:}
The goal of this problem is to fill the grid with 1 and 0. 
Rules:
\begin{description}
\item[1.] There have to be the same number of 1 and 0 in each line.
\item[2.] No more than two cells in a row can contain the same digit.
\item[3.] Each row and each column have to be unique.
\end{description}

\begin{table}[ht]
\centering
\begin{tabular}{|c|c|c|c|c|c|c|c|c|c|c|}
\hline 1 & 1 &   &   & 0 &   &   & 1 &   &   \\ 
\hline 1 &   & 1 &   &   &   & 0 &   &   & 1 \\ 
\hline   &   &   &   & 0 &   & 0 &   &   & 1 \\ 
\hline   & 0 &   & 1 &   & 1 &   &   &   &   \\ 
\hline   & 0 & 0 &   &   &   & 0 &   &   &   \\ 
\hline   &   &   &   &   &   &   & 0 &   &   \\ 
\hline 0 &   &   & 1 &   & 1 & 1 &   & 0 &   \\ 
\hline   &   & 0 &   &   &   & 1 &   &   &   \\ 
\hline 0 &   & 0 &   &   &   &   & 1 &   &   \\ 
\hline   & 1 &   &   & 1 &   & 1 &   &   & 0 \\
\hline
\end{tabular}
\end{table} 

Answer:
\begin{table}[ht]
\centering
\begin{tabular}{|c|c|c|c|c|c|c|c|c|c|c|}
\hline 1 & 1 & 0 & 1 & 0 & 0 & 1 & 1 & 0 & 0 \\ 
\hline 1 & 0 & 1 & 0 & 1 & 0 & 0 & 1 & 0 & 1 \\ 
\hline 0 & 1 & 1 & 0 & 0 & 1 & 0 & 0 & 1 & 1 \\ 
\hline 1 & 0 & 0 & 1 & 0 & 1 & 1 & 0 & 1 & 0 \\ 
\hline 1 & 0 & 0 & 1 & 1 & 0 & 0 & 1 & 0 & 1 \\ 
\hline 0 & 1 & 1 & 0 & 1 & 1 & 0 & 0 & 1 & 0 \\ 
\hline 0 & 0 & 1 & 1 & 0 & 1 & 1 & 0 & 0 & 1 \\ 
\hline 1 & 0 & 0 & 1 & 0 & 0 & 1 & 1 & 0 & 1 \\ 
\hline 0 & 1 & 0 & 0 & 1 & 1 & 0 & 1 & 1 & 0 \\ 
\hline 0 & 1 & 1 & 0 & 1 & 0 & 1 & 0 & 1 & 0 \\
\hline
\end{tabular}
\end{table} 


\subsection*{Brain teasers:}
\begin{description}
\item[A.] Rabbits breed like, well, rabbits if given the chance, and in a pattern closely resembling the Fibonacci sequence. If, at the beginning of each month each pair of reproductive age (at least two months old) rabbits produces 3 pairs of baby rabbits, how many rabbits pairs will there be after 5 months? 
%[19]

\item[B.] A string s is a supersequence of another string t if s contains t as a subsequence. Given two sequences ATCTGAT and TGCATA, what is the shortest possible supersequence so that both sequences are subsequences? 
%[ATGCATGAT]

\item[C.] Squaring the first several odd numbers reveals the following pattern: $3^2 = 8 + 1$, $5^2 = 24 + 1$, $7^2 = 48 + 1$. 8, 24, and 48 are all multiples of 8. Does this pattern hold for all squares of odd numbers? 
%[Yes; inductive proof shows this. Let base case be n (where n is an odd integer and n^2 mod 8 = 1). Does (n+2)^2 also yield 1 under modulo 8?]

\end{description}
\end{document}
