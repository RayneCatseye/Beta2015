\documentclass{article}

\usepackage{cwpuzzle}

\begin{document}
\section*{Puzzles}

\subsection*{Crossword:}

\begin{Puzzle}{14}{14} %
|*    |*    |*    |*    |*    |*    |*    |*    |*    |*    |*    |*    |[1]D |*    |.
|*    |*    |*    |*    |*    |*    |[2]K |*    |*    |*    |*    |*    |I    |*    |.
|*    |*    |*    |*    |*    |[3]P |A    |C    |K    |A    |G    |E    |S    |*    |.
|*    |*    |*    |*    |*    |*    |L    |*    |*    |*    |*    |*    |T    |*    |.
|*    |[4]K |*    |*    |*    |*    |I    |*    |[5]B |I    |N    |A    |R    |Y    |.
|*    |E    |*    |*    |*    |*    |*    |*    |A    |*    |*    |*    |I    |*    |.
|[6]G |R    |U    |B    |*    |[7]M |*    |*    |S    |*    |*    |*    |B    |*    |.
|*    |N    |*    |*    |*    |[8]A |R    |C    |H    |L    |I    |N    |U    |X    |.
|[9]D |E    |B    |I    |A    |N    |*    |*    |*    |*    |*    |*    |T    |*    |.
|*    |L    |*    |*    |*    |P    |*    |*    |*    |[10]S|*    |*    |I    |*    |.
|*    |*    |*    |[11]R|*    |A    |*    |[12]F|L    |U    |X    |B    |O    |X    |.
|*    |*    |*    |O    |*    |G    |*    |*    |*    |D    |*    |*    |N    |*    |.
|*    |[13]G|N    |O    |M    |E    |*    |*    |*    |O    |*    |*    |*    |*    |.
|*    |*    |*    |T    |*    |*    |*    |*    |*    |*    |*    |*    |*    |*    |.
\end{Puzzle}


\begin{PuzzleClues}{\textbf{Across}} %
\Clue{3}{PACKAGES}{"apt-get install"} %
\Clue{6}{GRUB}{Wriggly bootloader} %
\Clue{8}{ARCHLINUX}{Allows you to use PacMan whilst doing work} %
\Clue{9}{DEBIAN}{“The Universal Operating System”} %
\Clue{12}{FLUXBOX}{*box} %
\Clue{13}{GNOME}{Short, user friendly desktop environment} %
\end{PuzzleClues} %
%
\begin{PuzzleClues}{\textbf{Down}}
\Clue{1}{DISTRIBUTION}{Makes up an OS. Also, binomial \_\_\_\_.} %
\Clue{2}{KALI}{Pentesting distro} %
\Clue{4}{KERNEL}{At the centre of an OS} %
\Clue{5}{BASH}{Unix shell, good for percussive maintenance} %
\Clue{7}{MANPAGE}{Documentation galore, commandline accessed} %
\Clue{10}{SUDO}{\_\_\_\_ make me a sandwich.} %
\Clue{11}{ROOT}{Superuser, like carrots} %
\end{PuzzleClues}

\subsection*{Takuzu:}
The goal of this problem is to fill the grid with 1 and 0. 
Rules:
\begin{description}
\item[1.] There have to be the same number of 1 and 0 in each line.
\item[2.] No more than two cells in a row can contain the same digit.
\item[3.] Each row and each column have to be unique.
\end{description}

\begin{table}[ht]
\centering
\begin{tabular}{|c|c|c|c|c|c|c|c|c|c|c|}
\hline   & 1 &   & 0 &   &   & 1 &   &   &   \\ 
\hline   & 1 & 1 &   & 1 &   &   &   &   & 0 \\ 
\hline   &   &   &   &   &   &   &   &   & 0 \\ 
\hline 1 &   &   &   & 1 &   & 0 &   &   &   \\ 
\hline   &   &   & 0 &   &   &   & 1 &   &   \\ 
\hline   &   &   &   & 1 &   &   &   & 0 &   \\ 
\hline   &   &   & 0 &   & 0 &   &   &   &   \\ 
\hline 1 &   & 1 &   &   &   &   & 1 &   &   \\ 
\hline 1 &   &   &   & 1 &   &   &   & 0 &   \\ 
\hline   & 0 &   &   &   &   &   & 1 &   &   \\
\hline
\end{tabular}
\end{table} 

Answer:
\begin{table}[ht]
\centering
\begin{tabular}{|c|c|c|c|c|c|c|c|c|c|c|}
\hline 0 & 1 & 1 & 0 & 0 & 1 & 1 & 0 & 0 & 1 \\ 
\hline 0 & 1 & 1 & 0 & 1 & 0 & 0 & 1 & 1 & 0 \\ 
\hline 1 & 0 & 0 & 1 & 0 & 1 & 1 & 0 & 1 & 0 \\ 
\hline 1 & 0 & 0 & 1 & 1 & 0 & 0 & 1 & 0 & 1 \\ 
\hline 0 & 1 & 1 & 0 & 0 & 1 & 0 & 1 & 1 & 0 \\ 
\hline 1 & 0 & 0 & 1 & 1 & 0 & 1 & 0 & 0 & 1 \\ 
\hline 0 & 1 & 0 & 0 & 1 & 0 & 1 & 0 & 1 & 1 \\ 
\hline 1 & 0 & 1 & 1 & 0 & 1 & 0 & 1 & 0 & 0 \\ 
\hline 1 & 1 & 0 & 0 & 1 & 0 & 1 & 0 & 0 & 1 \\ 
\hline 0 & 0 & 1 & 1 & 0 & 1 & 0 & 1 & 1 & 0 \\
\hline
\end{tabular}
\end{table} 


\subsection*{Brain teasers:}
\begin{description}
\item[A.] Write a Linux kernel module, and stand-alone Makefile, that when loaded prints to the kernel debug log level, "Hello World!"  Be sure to make the module unloadable as well! (This problem was taken from: http://eudyptula-challenge.org/. Its great fun and I'd thoroughly recommend signing up for it.)

\item[B.] There is a pile of twelve coins, eleven of which are real and one of which is a counterfeit. The counterfeit coin will be either heavier or lighter than the other coins, which are all of equal weight. To find the counterfeit coin, you have a balance scale to place the coins on. In only THREE weighings, find which coin is counterfeit and whether it is heavier or lighter.  %[Damn, I love this problem]

\item[C.] Find the maximum area of a rectangle inscribed in a unit circle. %[2 units square]
\end{description}
\end{document}
