\documentclass{article}

\usepackage{cwpuzzle}

\begin{document}
\section*{Puzzles}

\subsection*{Crossword:}

% There's totally a regex crossword that goes here. If you manage to get it functioning in LaTeX I'll buy you whisky.

\subsection*{Takuzu:}
The goal of this problem is to fill the grid with 1 and 0. 
Rules:
\begin{description}
\item[1.] There have to be the same number of 1 and 0 in each line.
\item[2.] No more than two cells in a row can contain the same digit.
\item[3.] Each row and each column have to be unique.
\end{description}

\begin{table}[ht]
\centering
\begin{tabular}{|c|c|c|c|c|c|c|c|c|c|c|}
\hline   &   &   & 0 &   &   & 1 &   &   & 0 \\ 
\hline   & 1 & 1 &   &   &   &   & 0 &   &   \\ 
\hline   &   & 1 &   &   &   & 1 &   &   &   \\ 
\hline   &   &   &   &   &   &   &   & 1 & 1 \\ 
\hline   &   &   &   & 0 &   & 1 &   &   &   \\ 
\hline   &   &   &   &   & 0 &   &   & 0 &   \\ 
\hline   &   &   &   & 0 &   &   &   &   &   \\ 
\hline   &   & 1 & 1 &   &   & 0 & 1 &   & 1 \\ 
\hline 1 &   & 1 &   &   &   & 0 &   &   & 0 \\ 
\hline   &   &   &   &   &   &   &   &   &   \\
\hline
\end{tabular}
\end{table} 

Answer:
\begin{table}[ht]
\centering
\begin{tabular}{|c|c|c|c|c|c|c|c|c|c|c|}
\hline 1 & 1 & 0 & 0 & 1 & 0 & 1 & 1 & 0 & 0 \\ 
\hline 0 & 1 & 1 & 0 & 0 & 1 & 0 & 0 & 1 & 1 \\ 
\hline 1 & 0 & 1 & 1 & 0 & 0 & 1 & 1 & 0 & 0 \\ 
\hline 0 & 1 & 0 & 0 & 1 & 1 & 0 & 0 & 1 & 1 \\ 
\hline 1 & 0 & 0 & 1 & 0 & 1 & 1 & 0 & 1 & 0 \\ 
\hline 1 & 0 & 1 & 0 & 1 & 0 & 0 & 1 & 0 & 1 \\ 
\hline 0 & 1 & 0 & 1 & 0 & 1 & 1 & 0 & 1 & 0 \\ 
\hline 0 & 0 & 1 & 1 & 0 & 1 & 0 & 1 & 0 & 1 \\ 
\hline 1 & 0 & 1 & 0 & 1 & 0 & 0 & 1 & 1 & 0 \\ 
\hline 0 & 1 & 0 & 1 & 1 & 0 & 1 & 0 & 0 & 1 \\
\hline
\end{tabular}
\end{table} 


\subsection*{Brain teasers:}
\begin{description}
\item[A.] Whats the least number of changes required to turn the word “work” into “play”, where the only operations allowed are to add, remove or change a letter, and each intermediate word must be in English? %[6 steps: work, pork, perk, peak, peat, plat, play]

\item[B.] There's a king in the land of Binaria who’s planning on throwing an enormous party, and has amassed 1000 bottles of wine for the occasion. A would-be assassin breaks into the cellar where he’s storing this wine, and proceeds to poison one of the 1000 bottles, but gets away too quickly for the king's guard to see which one he poisoned or to catch him.
The king needs the remaining 999 safe bottles for his party in 4 weeks. Thankfully, the king has 10 servants who he considers disposable. The poison takes about 3 weeks to take effect, and any amount of it will kill whoever drinks it. How can he figure out which bottle was poisoned in time for the party?
Hint 1: The king will have to mix wine for this to work.
Hint 2: Its easier to think of this in binary numbers.
%[The king assigns each servant a number from 1-10. The king assigns each bottle a number from 0-999. When he labels them, though, he writes the number on the bottle in binary with ten digits, like this: 0: 000000000 1: 000000001 2: 000000010 3: 000000011 4: 000000100 5: 000000101 ... 999: 1111100111 and so on.
%Now, each servant takes a small sip from every bottle where the servant's number equals 1 in the binary number on the bottle. So, the 1st servant drinks from every other bottle. The second servant drinks from bottles 2, 3, 6, 7, 10, 11, etc.
%Then based on the combination of servants that die, he can identify the poisoned bottle. For example, if none of them die, the 0th bottle was poisoned because none of them drank from it. If only servant 1 dies, then bottle 1 was poisoned, because he's the only person who drank from it. Finally, if servants 1, 2, 3, 6, 7, 8, 9, and 10 die, then the 999th bottle was poisoned ]

\end{description}
\end{document}
