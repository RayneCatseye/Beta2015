\documentclass{article}

\usepackage{cwpuzzle}

\begin{document}
\section*{Puzzles}

\subsection*{Crossword:}

\begin{Puzzle}{18}{18} %
|*    |*    |*    |*    |*    |*    |*    |*    |[1]C |*    |*    |*    |*    |*    |*    |*    |*    |*    |.
|*    |*    |[2]K |*    |*    |*    |*    |*    |R    |*    |[3]C |*    |*    |*    |*    |*    |*    |*    |.
|[4]F |R    |E    |Q    |U    |E    |N    |C    |Y    |*    |A    |*    |*    |*    |*    |*    |*    |*    |.
|*    |*    |R    |*    |*    |*    |*    |*    |P    |*    |E    |*    |*    |*    |*    |*    |*    |*    |.
|*    |*    |C    |*    |*    |*    |*    |*    |T    |*    |S    |*    |*    |*    |*    |*    |*    |*    |.
|*    |*    |K    |*    |*    |*    |*    |*    |A    |*    |A    |*    |*    |*    |*    |*    |*    |[5]C |.
|*    |*    |H    |*    |[6]V |I    |G    |E    |N    |E    |R    |E    |C    |I    |P    |H    |E    |R    |.
|*    |*    |O    |*    |*    |*    |*    |*    |A    |*    |C    |*    |*    |*    |*    |*    |*    |Y    |.
|*    |*    |F    |*    |[7]C |*    |*    |[8]P |L    |A    |I    |N    |T    |E    |X    |T    |*    |P    |.
|*    |*    |F    |*    |O    |*    |*    |*    |Y    |*    |P    |*    |*    |*    |*    |*    |*    |T    |.
|*    |*    |*    |*    |L    |*    |*    |*    |S    |*    |H    |*    |[9]l |*    |*    |*    |*    |O    |.
|*    |*    |*    |*    |[10]O|N    |[11]E|T    |I    |M    |E    |P    |A    |D    |*    |*    |*    |G    |.
|*    |*    |*    |*    |S    |*    |N    |*    |S    |*    |R    |*    |N    |*    |*    |*    |*    |R    |.
|*    |*    |*    |*    |S    |*    |T    |*    |*    |*    |*    |*    |D    |*    |*    |[12]K|*    |A    |.
|*    |*    |*    |*    |U    |*    |R    |*    |*    |*    |[13]S|H    |A    |O    |N    |E    |*    |P    |.
|*    |*    |*    |*    |S    |*    |O    |*    |*    |*    |*    |*    |U    |*    |*    |Y    |*    |H    |.
|*    |*    |*    |*    |*    |*    |P    |*    |[14]C|I    |P    |H    |E    |R    |*    |*    |*    |Y    |.
|*    |*    |*    |*    |*    |*    |Y    |*    |*    |*    |*    |*    |R    |*    |*    |*    |*    |*    |.
\end{Puzzle}


\begin{PuzzleClues}{\textbf{Across}} %
\Clue{4}{FREQUENCY}{How often letters occur} %
\Clue{6}{VIGNERECIPHER}{Multiples of [caesarciphers], in a row.} %
\Clue{8}{PLAINTEXT}{Pre-encrypted text, much less exciting} %
\Clue{10}{ONETIMEPAD}{Unbreakable if things are really random} %
\Clue{13}{SHAONE}{160-bit hash function} %
\Clue{14}{CIPHER}{Replaces one with another} %
\end{PuzzleClues} %
%
\begin{PuzzleClues}{\textbf{Down}}
\Clue{1}{CRYPTANALYSIS}{Deciphering codes (not necessarily perl)} %
\Clue{2}{KERCKHOFF}{Key, not algorithm, secrecy} %
\Clue{3}{CAESARCIPHER}{rot13}
\Clue{5}{CRYPTOGRAPHY}{Hidden writing, but in Greek} %
\Clue{7}{COLOSSUS}{Cracked the Lorenz cipher} %
\Clue{9}{LANDAUER}{128-bit theoretical limit, but people} %
\Clue{11}{ENTROPY}{Shannon measure of coding efficiency} %
\Clue{12}{KEY}{Public and private} %
\end{PuzzleClues}

\subsection*{Takuzu:}
The goal of this problem is to fill the grid with 1 and 0. 
Rules:
\begin{description}
\item[1.] There have to be the same number of 1 and 0 in each line.
\item[2.] No more than two cells in a row can contain the same digit.
\item[3.] Each row and each column have to be unique.
\end{description}

\begin{table}[ht]
\centering
\begin{tabular}{|c|c|c|c|c|c|c|c|c|c|c|}
\hline 1 &   &   & 0 &   & 1 & 0 &   & 0 &   \\ 
\hline   &   &   &   &   &   &   & 1 &   &   \\ 
\hline   & 0 &   & 1 &   &   &   &   &   &   \\ 
\hline   &   &   &   &   & 1 &   &   & 0 &   \\ 
\hline 1 &   & 0 &   &   &   & 0 &   &   &   \\ 
\hline   &   &   &   & 1 &   &   &   &   & 1 \\ 
\hline   & 1 &   &   &   & 1 &   &   & 1 &   \\ 
\hline 0 & 1 &   &   & 0 &   & 1 &   &   & 1 \\ 
\hline   &   &   & 0 &   &   &   &   & 1 & 1 \\ 
\hline 0 &   &   & 0 &   & 0 &   &   & 0 &   \\
\hline
\end{tabular}
\end{table} 

Answer:
\begin{table}[ht]
\centering
\begin{tabular}{|c|c|c|c|c|c|c|c|c|c|c|}
\hline 1 & 0 & 1 & 0 & 0 & 1 & 0 & 1 & 0 & 1 \\ 
\hline 0 & 1 & 0 & 1 & 1 & 0 & 0 & 1 & 1 & 0 \\ 
\hline 1 & 0 & 0 & 1 & 1 & 0 & 1 & 0 & 1 & 0 \\ 
\hline 0 & 1 & 1 & 0 & 0 & 1 & 1 & 0 & 0 & 1 \\ 
\hline 1 & 0 & 0 & 1 & 0 & 1 & 0 & 1 & 1 & 0 \\ 
\hline 1 & 0 & 0 & 1 & 1 & 0 & 1 & 0 & 0 & 1 \\ 
\hline 0 & 1 & 1 & 0 & 1 & 1 & 0 & 0 & 1 & 0 \\ 
\hline 0 & 1 & 0 & 1 & 0 & 0 & 1 & 1 & 0 & 1 \\ 
\hline 1 & 0 & 1 & 0 & 0 & 1 & 0 & 0 & 1 & 1 \\ 
\hline 0 & 1 & 1 & 0 & 1 & 0 & 1 & 1 & 0 & 0 \\
\hline
\end{tabular}
\end{table} 


\subsection*{Brain teasers:}
\begin{description}
\item[A.] Guvf vf gur bayl Pnrfne Pvcure gung vf vgf bja vairefr. Juvpu bar vf vg? 
%[ROT13]

\item[B.] Most books have an ISBN number. e.g.  the ISBN number of “Applied Cryptography” is 0471117099 .  Let's multiply these digits by the numbers ten to one:  10*0+9*4+8*7+7*1+6*1+5*1+4*7+3*0+2*9+1*9 = 165 = 15*11.  As you can see, the sum is divisible by 11. In fact, this property holds true for all ISBNs.  
The number  0100000001 cannot be an ISBN. If we know that one digit is wrong, how many possible ISBNs could it have originally been? 
%[9: 0600000001, 0170000001, 0108000001, 0100200001, 0100090001,  0100003001, 0100000401, 0100000061, or 0100000002]

\item[C.] In this puzzle, each letter stands for unique digit that makes the arithmetic equation true. What are the values of the letters in: SEVEN + SEVEN + SIX = TWENTY
%[68782 + 68782 + 650 = 138214]


\end{description}
\end{document}
