\documentclass{article}

\usepackage{cwpuzzle}

\begin{document}
\section*{Puzzles}

\subsection*{Crossword:}

\begin{Puzzle}{16}{19} %
|[1]B |*    |*    |*    |*    |[2]T |*    |*    |*    |*    |*    |*    |*    |*    |*    |*    |.
|U    |*    |*    |*    |*    |E    |*    |*    |*    |*    |*    |*    |*    |*    |*    |*    |.
|[3]S |E    |R    |V    |E    |R    |*    |*    |*    |*    |*    |*    |[4]M |*    |*    |*    |.
|*    |*    |*    |*    |*    |[5]M |O    |T    |H    |E    |R    |B    |O    |A    |R    |D    |.
|*    |*    |*    |*    |*    |I    |*    |*    |*    |*    |*    |*    |B    |*    |*    |*    |.
|*    |*    |*    |*    |[6]I |N    |S    |T    |R    |U    |C    |T    |I    |O    |N    |*    |.
|*    |*    |*    |*    |*    |A    |*    |*    |*    |*    |*    |*    |L    |*    |*    |*    |.
|*    |*    |*    |*    |*    |L    |*    |[7]C |H    |I    |P    |S    |E    |T    |*    |*    |.
|*    |*    |*    |*    |*    |*    |*    |*    |*    |*    |*    |*    |     |*    |*    |*    |.
|*    |*    |[8]R |A    |I    |D    |*    |*    |*    |*    |*    |*    |P    |*    |*    |*    |.
|*    |*    |*    |*    |*    |*    |*    |*    |*    |*    |[9]M |*    |H    |*    |*    |*    |.
|*    |*    |*    |[10]A|C    |C    |U    |M    |U    |L    |A    |T    |O    |R    |*    |*    |.
|*    |*    |*    |*    |*    |*    |*    |*    |*    |*    |I    |*    |N    |*    |*    |*    |.
|*    |*    |*    |*    |*    |*    |*    |*    |*    |*    |N    |*    |E    |*    |*    |*    |.
|*    |*    |*    |*    |*    |*    |*    |*    |*    |*    |F    |*    |*    |*    |*    |*    |.
|*    |*    |*    |*    |[11]S|O    |F    |T    |W    |A    |R    |E    |*    |*    |*    |*    |.
|*    |*    |*    |*    |*    |*    |*    |*    |*    |*    |A    |*    |*    |*    |*    |*    |.
|*    |*    |*    |*    |*    |*    |*    |*    |*    |*    |M    |*    |*    |*    |*    |*    |.
|*    |*    |*    |[12]F|I    |R    |M    |W    |A    |R    |E    |*    |*    |*    |*    |*    |.
\end{Puzzle}


\begin{PuzzleClues}{\textbf{Across}} %
\Clue{3}{SERVER}{Computer used to provide services to clients.} %
\Clue{5}{MOTHERBOARD}{Parental circuit board} %
\Clue{6}{INSTRUCTION}{Group of several bits containing an operation code} %
\Clue{7}{CHIPSET}{Group of integrated circuits, or chips, designed to work together} %
\Clue{8}{RAID}{Data storage scheme across multiple hard disk drives} %
\Clue{10}{ACCUMULATOR}{Register in a CPU where intermediate arithmetic and logic results are stored.} %
\Clue{11}{SOFTWARE}{Computer programs and other kinds of information read and written by computers} %
\Clue{12}{FIRMWARE}{Fixed programs and data that internally control various electronic devices} %
\end{PuzzleClues} %
%
\begin{PuzzleClues}{\textbf{Down}}
\Clue{1}{BUS}{subsystem that transfers data between computer components inside a computer or between computers} %
\Clue{2}{TERMINAL}{device that is used for entering data into, and displaying data from, a computer} %
\Clue{4}{MOBILE PHONE}{Pocket sized computer, with apps!} %
\Clue{9}{MAINFRAME}{Computers used for bulk data processing, no clouds involved.} %
\end{PuzzleClues}

\subsection*{Takuzu:}
The goal of this problem is to fill the grid with 1 and 0. 
Rules:
\begin{description}
\item[1.] There have to be the same number of 1 and 0 in each line.
\item[2.] No more than two cells in a row can contain the same digit.
\item[3.] Each row and each column have to be unique.
\end{description}

\begin{table}[ht]
\centering
\begin{tabular}{|c|c|c|c|c|c|c|c|c|c|c|}
\hline   & 1 &   & 0 &   &   & 1 &   &   &   \\ 
\hline   & 1 & 1 &   & 1 &   &   &   &   & 0 \\ 
\hline   &   &   &   &   &   &   &   &   & 0 \\ 
\hline 1 &   &   &   & 1 &   & 0 &   &   &   \\ 
\hline   &   &   & 0 &   &   &   & 1 &   &   \\ 
\hline   &   &   &   & 1 &   &   &   & 0 &   \\ 
\hline   &   &   & 0 &   & 0 &   &   &   &   \\ 
\hline 1 &   & 1 &   &   &   &   & 1 &   &   \\ 
\hline 1 &   &   &   & 1 &   &   &   & 0 &   \\ 
\hline   & 0 &   &   &   &   &   & 1 &   &   \\
\hline
\end{tabular}
\end{table} 

Answer:
\begin{table}[ht]
\centering
\begin{tabular}{|c|c|c|c|c|c|c|c|c|c|c|}
\hline 0 & 1 & 1 & 0 & 0 & 1 & 1 & 0 & 0 & 1 \\ 
\hline 0 & 1 & 1 & 0 & 1 & 0 & 0 & 1 & 1 & 0 \\ 
\hline 1 & 0 & 0 & 1 & 0 & 1 & 1 & 0 & 1 & 0 \\ 
\hline 1 & 0 & 0 & 1 & 1 & 0 & 0 & 1 & 0 & 1 \\ 
\hline 0 & 1 & 1 & 0 & 0 & 1 & 0 & 1 & 1 & 0 \\ 
\hline 1 & 0 & 0 & 1 & 1 & 0 & 1 & 0 & 0 & 1 \\ 
\hline 0 & 1 & 0 & 0 & 1 & 0 & 1 & 0 & 1 & 1 \\ 
\hline 1 & 0 & 1 & 1 & 0 & 1 & 0 & 1 & 0 & 0 \\ 
\hline 1 & 1 & 0 & 0 & 1 & 0 & 1 & 0 & 0 & 1 \\ 
\hline 0 & 0 & 1 & 1 & 0 & 1 & 0 & 1 & 1 & 0 \\
\hline
\end{tabular}
\end{table} 


\subsection*{Brain teasers:}
\begin{description}
\item[A.] Using a 5 litre jug, a 3 litre jug, and a hose, can you measure 1 litre of water? % Fill the 3L and put it into the 5L jug. Fill the 3L jug again, then pour as much as will fit into the 5L jug. The remainder in the 3L jug will be 1L.

\item[B.] How about 4 litres? % As above to measure 1L; place this in the 5L jug. Then fill the 3L jug and add this to the 1L.

\item[C.] You are given 12 coins of equal size and shape; one of them is a fake. We do  not know if it is heavier or lighter than the others, but want to find out. Can you determine which one is a fake and if it is lighter or heavier by weighing coins on a balance scale a maximum of three times? % Remember that not on the scale is a valid option.
\end{description}
\end{document}
