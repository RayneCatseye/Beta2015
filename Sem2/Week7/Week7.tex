\documentclass{article}

\usepackage{cwpuzzle}

\begin{document}
\section*{Puzzles}

\subsection*{Crossword:}

\begin{Puzzle}{15}{15} %
|*    |*    |*    |[1]A |L    |I    |A    |[2]S |*    |*    |*    |*    |*    |*    |*    |.
|[3]C |*    |*    |*    |*    |*    |*    |T    |*    |*    |*    |[4]C |*    |*    |*    |.
|M    |*    |*    |*    |*    |*    |*    |E    |*    |[5]P |*    |H    |*    |*    |*    |.
|[6]A |S    |[7]C |I    |I    |Q    |U    |A    |R    |I    |U    |M    |*    |*    |*    |.
|T    |*    |O    |*    |*    |*    |*    |M    |*    |P    |*    |O    |*    |[8]R |*    |.
|R    |*    |W    |*    |*    |*    |*    |L    |*    |E    |*    |D    |*    |S    |*    |.
|I    |*    |S    |*    |*    |*    |*    |O    |*    |*    |*    |*    |*    |Y    |*    |.
|X    |*    |A    |*    |*    |*    |[9]S |C    |R    |I    |P    |T    |I    |N    |G    |.
|*    |*    |Y    |*    |*    |*    |*    |O    |*    |*    |*    |*    |*    |C    |*    |.
|*    |*    |*    |*    |*    |*    |*    |M    |*    |*    |*    |*    |*    |*    |*    |.
|*    |*    |*    |[10]P|Y    |T    |H    |O    |N    |*    |*    |*    |*    |*    |*    |.
|*    |*    |*    |*    |*    |*    |*    |T    |*    |*    |*    |*    |*    |*    |*    |.
|*    |*    |*    |*    |*    |*    |[11]P|I    |N    |G    |*    |*    |*    |*    |*    |.
|*    |*    |*    |*    |*    |*    |*    |V    |*    |*    |*    |*    |*    |*    |*    |.
|*    |*    |*    |*    |*    |[12]G|R    |E    |P    |*    |*    |*    |*    |*    |*    |.
\end{Puzzle}


\begin{PuzzleClues}{\textbf{Across}} %
\Clue{1}{ALIAS}{Type one thing, mean another} %
\Clue{6}{ASCIIQUARIUM}{Make fish to watch you while you work} %
\Clue{9}{SCRIPTING}{How one automates on the commandline} %
\Clue{10}{PYTHON}{Common scripting language} %
\Clue{11}{PING}{Sonar for the web} %
\Clue{12}{GREP}{Type a string, get a line} %
\end{PuzzleClues} %
%
\begin{PuzzleClues}{\textbf{Down}}
\Clue{2}{STEAMLOCOMOTIVE}{sl} %
\Clue{3}{CMATRIX}{The matrix right there in your terminal} %
\Clue{4}{CHMOD}{Changes permissions}
\Clue{5}{PIPE}{Carrying data, not water} %
\Clue{7}{COWSAY}{Cows with strings} %
\Clue{8}{RSYNC}{Data sync (not sink)} %
\end{PuzzleClues}

\subsection*{Takuzu:}
The goal of this problem is to fill the grid with 1 and 0. 
Rules:
\begin{description}
\item[1.] There have to be the same number of 1 and 0 in each line.
\item[2.] No more than two cells in a row can contain the same digit.
\item[3.] Each row and each column have to be unique.
\end{description}

\begin{table}[ht]
\centering
\begin{tabular}{|c|c|c|c|c|c|c|c|c|c|c|}
\hline   &   &   & 0 &   &   & 1 &   &   & 0 \\ 
\hline   & 1 & 1 &   &   &   &   & 0 &   &   \\ 
\hline   &   & 1 &   &   &   & 1 &   &   &   \\ 
\hline   &   &   &   &   &   &   &   & 1 & 1 \\ 
\hline   &   &   &   & 0 &   & 1 &   &   &   \\ 
\hline   &   &   &   &   & 0 &   &   & 0 &   \\ 
\hline   &   &   &   & 0 &   &   &   &   &   \\ 
\hline   &   & 1 & 1 &   &   & 0 & 1 &   & 1 \\ 
\hline 1 &   & 1 &   &   &   & 0 &   &   & 0 \\ 
\hline   &   &   &   &   &   &   &   &   &   \\
\hline
\end{tabular}
\end{table} 

Answer:
\begin{table}[ht]
\centering
\begin{tabular}{|c|c|c|c|c|c|c|c|c|c|c|}
\hline 1 & 1 & 0 & 0 & 1 & 0 & 1 & 1 & 0 & 0 \\ 
\hline 0 & 1 & 1 & 0 & 0 & 1 & 0 & 0 & 1 & 1 \\ 
\hline 1 & 0 & 1 & 1 & 0 & 0 & 1 & 1 & 0 & 0 \\ 
\hline 0 & 1 & 0 & 0 & 1 & 1 & 0 & 0 & 1 & 1 \\ 
\hline 1 & 0 & 0 & 1 & 0 & 1 & 1 & 0 & 1 & 0 \\ 
\hline 1 & 0 & 1 & 0 & 1 & 0 & 0 & 1 & 0 & 1 \\ 
\hline 0 & 1 & 0 & 1 & 0 & 1 & 1 & 0 & 1 & 0 \\ 
\hline 0 & 0 & 1 & 1 & 0 & 1 & 0 & 1 & 0 & 1 \\ 
\hline 1 & 0 & 1 & 0 & 1 & 0 & 0 & 1 & 1 & 0 \\ 
\hline 0 & 1 & 0 & 1 & 1 & 0 & 1 & 0 & 0 & 1 \\
\hline
\end{tabular}
\end{table} 


\subsection*{Brain teasers:}
\begin{description}
\item[A.] How would you determine if a given binary tree is valid or not? % Recursive is the way to go with this; remember that left is always smaller than the parent, and right is larger.

\item[B.] Break some C programming mishaps (and get some commandline skills) with a wargame: http://overthewire.org/wargames/manpage/

\end{description}
\end{document}
