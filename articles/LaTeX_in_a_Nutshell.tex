\documentclass{article}

\begin{document}
\section*{LaTeX: a document preparation system}
One thing that you’ll likely come across (or certainly come across, if you do algos) during your time in computing is LaTeX. Like so many other things in [industry|academia|computing], there is no mention of it until there is an expectation that you already know the program. So what is LaTeX? 

At its core, LaTeX is a program that allows you to create documents for other people to read (in fact, this document it rendered in LaTeX!). In that respect it is similar to Word, but thats about the only way. Unlike Word or its analogues, which require a specific program to be run, LaTeX document generation generally starts with any old text editor (from vim, to gedit to something more specialised), which is used to edit a LaTeX source file (a .tex file). This source file is then run through the LaTeX program to convert it to a document format, e.g. PDF. At this stage you're ready to send the finished product to whoever you like (or reformat it if you realise that you’ve made a terrible mistake).

The biggest difference between Word and LaTeX is the presence or absence of a GUI. Word is all about making things pretty, and letting you see it, as soon as you write them, whereas in LaTeX you write things, and only once you’ve converted it do you see the finished product. This is a mixed blessing; it allows LaTeX to be more flexible (want to change the spacing? Change the flags for the conversion process), but less obviously usable to an audience unused to it. The various pros and cons are discussed later in this article.

\section*{What's in a name?}

LaTeX’s name is, really, one of the most frustrating things about it. As well as making it hard to Google without bringing up results full of fetish gear, the pronunciation will drive you crazy while you’re getting used to it (though I’ll talk more about that later). But at least it isn’t called Elm like everything else seems to be. 

\textbf{Typography:} Using a generally infeasible combination of capitalization, font sizing, and baseline offsetting, LaTeX is often (and officially) written: \LaTeX  There is a reason for this bizarre arrangement of letters, I promise; its in part to distinguish it from the aforementioned rubber-like substance, but mostly to show off what is possible in LaTeX. If this were a Word document, that would have to be an embedded image; so much more frustrating to work with. 

\textbf{Pronunciation:} This is a make or break thing in certain groups (I know, its absolutely stupid); the generally accepted pronunciation is "LAH-tek". Whatever you do, don't pronounce the second syllable "teks" or call it latex. It is not a rubbery substance, and Knuth may jump down from the ceiling and glare at you.

\textbf{Etymology:} Like so many other things, it was Donald Knuth who first created "TEX" (the three characters actually being uppercase Greek tau, epsilon and chi). This Greek forms the root of English words like "technical" and "technique", and the choice to group TeX with this was a conscious one. Later, Leslie Lamport built "LaTeX" on top of Tex, presumably prepending “La” because “Lamport’s TeX” was too long.

\section*{Pros of LaTeX}
So now we have a brief idea of what LaTeX is, why do people use it?
\begin{itemize}
\item \textbf{Superior typogaphic quality:} A great reason to use LaTeX (aside from all the other pros in this list) is that it produces beautiful documents. You can fine tune your line spacings, the inbuilt fonts are professional The main reason I use LaTeX is it produces output that is, typographically, far better than any of the alternatives. LaTeX has excellent built-in fonts, good algorithms for automatic spacing, and the ability to fine-tune the spacing arbitrarily. Bad typography gives a bad first impression, and reflects poorly on the content of a document.
\item \textbf{Portability:} LaTeX is the single most portable document creation system of them all, running on virtually any operating system in existence (maybe not DOS? Need to check). Comparatively, MS Word only works on Windows and Mac, and even OpenOffice runs on all Unix breeds.
\item \textbf{Compatibility with revision control:} Because .tex files are plain text, you can use git, diff and a variety of other tools to look at the change history. Far easier than the complex systems (or lack thereof!) used elsewhere. 
\item \textbf{Document longevity:} LaTeX documents are functionally timeless: those written 10 years ago still work and still produce (mostly) the same output as they did at the start. In contrast, Word documents are typically useful only for 3-4 years before they stop working properly on new versions.
\item \textbf{Mathematical typesetting:} Unlike other systems, maths can be input inline, and doesn’t skew line width formatting. Beautiful.
\item \textbf{Macros:} LaTeX lets me define macros, canned sequences of text and/or markup, that I can then use repeatedly. It's much better than copy+paste since it can be changed by changing just the definition, plus I don't have to find the original each time.
\item \textbf{Peer pressure:} In academic publishing (my data is, admittedly, limited to computer science and physics) you’re often not taken seriously unless you use LaTeX. As far as incentives go to use LaTeX, that’s often the strongest of them all. 
\end{itemize}

\section*{Cons of LaTeX}
So given all of the great things about LaTeX, why doesn’t everyone use it?
\begin{itemize}
\item \textbf{Fragmented programs:} Writing a document with LaTeX means using an editor, LaTeX itself, a document previewer, and usually a few other assorted programs. In contrast, Word and other such programs are self contained.
\item \textbf{Difficulty knowing/remembering commands:} Learning markup commands takes time, and can initially be a really frustrating endeavour (like learning any new programming language). I’d recommend using a helpful GUI to start, before moving on to vim or similar (if you want, some of the GUIs are great to stay o forever).
\item \textbf{Preview delay:} There’s a delay between typing something in the editor and seeing the result in the document previewer; depending on how often you preview, this can be a long or short gap and more or less frustrating.
\item \textbf{Syntax errors:} Like in all programming languages, it’s absolutely possible to create a .tex file that LaTeX will reject, complaining of a syntax error (and aren’t they just your favourite). Unfortunately, the errors are often cryptic, and take some headbanging to fix. There’s definitely a learning curve in dealing with them!
\end{itemize}

Overall, I’d recommend using LaTeX, because once you’re over the initial frustration with it, it’s a highly useful skill to have. Programs such as TeXstudio (available for all OSes) are a nice combination of helpful but not coddling, and take a lot of the more frustrating guesswork out of the process. So write an assignment in LaTeX this sem, or convert an old project into LaTeX as an experiment. Though it will probably be frustrating, it will (probably) wind up being fun.
\end{document}